\documentclass[
	12pt,
	paper=A4,
	oneside,
	draft
]{scrreprt}

% Кодировка, шрифты и языки
\usepackage{cmap}
\usepackage[T2A]{fontenc}
\usepackage[utf8]{inputenc}
\usepackage[english,russian,ukrainian]{babel}

% Лист и структура
\usepackage[a4paper, top=2cm, bottom=2cm, left=3cm, right=1cm]{geometry}
\usepackage{indentfirst}
\usepackage{enumitem}
%\usepackage{multicol}

% Разное
\usepackage{datetime}
\usepackage{ifthen}
\usepackage{lipsum}

% Абзацный отступ
\parindent=1.25cm


% Настройка колонтитулов, в т.ч. - номера страницы
\usepackage{fancyhdr}
\fancyhf{} % очистка текущих значений
\fancyhead[R]{\thepage} % установка верхнего колонтитула
\renewcommand{\headrulewidth}{0pt} % убрать разделительную линию
\pagestyle{fancy}

% Pictures (plots ect)
\usepackage{pgfplots} 
%% Export to CSV
%\usepackage{filecontents}
%\begin{filecontents*}{data.csv}
%n,s1,s2,s3,s4
%5,0.00001168,0.00001121,0.00002050,0.00001121
%
%\end{filecontents*}
\usepackage[raggedright]{titlesec}
% Управление содержанием
\usepackage{titletoc}
\titleformat{\chapter}[display] % раздел
	{\filcenter}	% центровать
    %{\MakeUppercase{\chaptertitlename} \thechapter}	% РОЗДІЛ Х 
    {}
    {0pt} % отступ после
	{\MakeUppercase}	% название капсом
	{}
\assignpagestyle{\chapter}{fancy}
\titlespacing{\chapter} % указуємо, що модифікуємо саме розділ
	{0ex} % відступ зліва
	{-30pt} % відступ зверху 
	{8pt} % відступ знизу
\titlecontents{chapter}
	[0ex] %
	{}
	{\MakeUppercase{\chaptername\ \thecontentslabel {}\quad}\MakeUppercase}
	{}
	{\dotfill\contentspage}    

%	ПІДРОЗДІЛ
\titleformat{\section}[block]
    {\normalsize}
    {\thesection\quad}
    {1em}
    {}
\titlespacing{\section}
	{\parindent}
	{0pt}
	{0pt}
\titlecontents{section}
	[5ex]
	{}
	{\thecontentslabel\quad}
	{}
	{\dotfill\contentspage}    

\usepackage{pgfplotstable}
\usepackage{csvsimple}
\usepackage{longtable}
\begin{document}
\chapter{Построение и анализ алгоритмов}
\section{Анализ алгоритмов сортировки}
\thispagestyle{empty}

Сортировка простыми обменами, сортировка пузырьком\,--\,простой алгоритм сортировки. Для понимания и реализации этот алгоритм — простейший, но эффективен он лишь для небольших массивов. Сложность алгоритма: $ O(n^{2})$. В то же время метод сортировки обменами лежит в основе некоторых более совершенных алгоритмов, таких как шейкерная сортировка, пирамидальная сортировка и быстрая сортировка.

Сортировка вставками\,--\,достаточно простой алгоритм. Основным преимуществом алгоритма сортировки вставками является возможность сортировать массив по мере его получения.

Сортировка выбором\,--\,алгоритм сортировки. Может быть как устойчивый, так и неустойчивый. На массиве из $n$ элементов имеет время выполнения в худшем, среднем и лучшем случае $ O(n^{2})$, предполагая что сравнения делаются за постоянное время.

Быстрая сортировка, сортировка Хоара\,--\,один из самых быстрых известных универсальных алгоритмов сортировки массивов: в среднем $O(n\log n)$ обменов при упорядочении $n$ элементов; из-за наличия ряда недостатков на практике обычно используется с некоторыми доработками.

%\pgfplotstabletypeset[col sep=comma,
%     columns={n,s1,s2,s3,s4},
%    ]{data.csv}

\csvloop{
  file=data.csv,
  no head,
  column count=5,% could may removed for 10 or less columns
  before reading=\begin{longtable}{l|l|l|l|l}\caption{Время выполнения сортировок\label{tab:Sort}}\endlastfoot,
  command={\csviffirstrow%
    {N & select (1) & select (2) & quick & bubble}
    {\csvcoli & \csvcolii & \csvcoliii & \csvcoliv & \csvcolv}%
  },
  late after line=\\,
  late after first line=\\\hline\endhead,
  late after last line=,
  after reading=\end{longtable}
}
В таблице \ref{tab:Sort} приведены результаты выполнения реализаций алгоритмов сортироки на языке Python в зависимости от размера сортируемого массива, где

\textit{N}~--- количество элементов в массиве (его размер);

\textit{select (1)}~--- сортировка выбором с использованием конструкции \textit{while};

\textit{select (2)}~--- сортировка выбором с использованием конструкции \textit{for... in range()};

\textit{quick}~--- быстрая сортировка;

\textit{bubble}~--- сортировка пузырьком.
\bigskip

Из данных в этой таблице построены графики на рисунке \ref{fig:Sort}, из которых очевидно, что время выполнения сортировки пузырьком растет существенно быстрее остальных сортировок.

\begin{figure}[hb]
\centering
\begin{tikzpicture}
\begin{axis}[
	xlabel={размер массива, \textit{n}},
    ylabel={время сортировки, \textit{s}},
    scaled x ticks = false,
    legend style={at={(0.0,1.0)}, anchor=north west}]

	\addplot table [y=s1, x=n, col sep=comma] {data.csv};
	\addplot table [y=s2, x=n, col sep=comma] {data.csv};
	\addplot table [y=s3, x=n, col sep=comma] {data.csv};
	\addplot table [y=s4, x=n, col sep=comma] {data.csv};
	
	\addlegendentry{select (1)}
	\addlegendentry{select (2)}
	\addlegendentry{quick}
	\addlegendentry{bubble}
\end{axis}
\end{tikzpicture}
\caption{Зависимость времени выполнения алгоритмов от размера массива} \label{fig:Sort}
\end{figure}

При этом можно сделать вывод, что реализация алгоритма с участием конструкции \textit{while} быстрее, чем аналогичная с использованием конструкции \textit{for... in range()}. Последняя, при своей работе, создает множество, перебирая все его элементы, таким образом являясь аналогом \textit{foreach} из некоторых других языков, например PHP.

\begin{figure}[hb]
\centering
\begin{tikzpicture}
\begin{axis}[
	xlabel={размер массива, \textit{n}},
    ylabel={время сортировки, \textit{ms}},
    scaled x ticks = false,
    ytick scale label code/.code={},
    legend style={at={(0.0,1.0)}, anchor=north west}]

	\addplot table [y=s1, x=n, col sep=comma] {data2.csv};
	\addplot table [y=s2, x=n, col sep=comma] {data2.csv};
	\addplot table [y=s3, x=n, col sep=comma] {data2.csv};
	\addplot table [y=s4, x=n, col sep=comma] {data2.csv};
	
	\addlegendentry{select (1)}
	\addlegendentry{select (2)}
	\addlegendentry{quick}
	\addlegendentry{bubble}
\end{axis}
\end{tikzpicture}
\caption{Зависимость времени выполнения от размера для небольших массивов} \label{fig:SortSmall}
\end{figure}

Из представленных графиков можно сделать вывод, что сортировку пузырьком целесообразно использовать лишь для небольших массивов (порядка десятков элементов). 

В то же время видно, что представленная реализация алгоритма быстрой сортировки проигрывает во времени алгоритму сортировки выбором на небольших массивах. Ведь, несмотря на меньшую алгоритмичискую сложность ($O(n\log n)$ против $O(n^2)$), при её оценке не учитывается константа $c$, которая играет роль при небольших $n$.
\vfill
\begin{flushright}
Сенчишен Д.А.
\end{flushright}

\end{document}
