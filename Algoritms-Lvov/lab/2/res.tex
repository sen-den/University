\documentclass[
	12pt,
	paper=A4,
	oneside,
	draft
]{scrreprt}

% Кодировка, шрифты и языки
\usepackage{cmap}
\usepackage[T2A]{fontenc}
\usepackage[utf8]{inputenc}
\usepackage[english,russian,ukrainian]{babel}

% Лист и структура
\usepackage[a4paper, top=2cm, bottom=2cm, left=3cm, right=1cm]{geometry}
\usepackage{indentfirst}
\usepackage{enumitem}
%\usepackage{multicol}

% Разное
\usepackage{datetime}
\usepackage{ifthen}
\usepackage{lipsum}

% Абзацный отступ
\parindent=1.25cm


% Настройка колонтитулов, в т.ч. - номера страницы
\usepackage{fancyhdr}
\fancyhf{} % очистка текущих значений
\fancyhead[R]{\thepage} % установка верхнего колонтитула
\renewcommand{\headrulewidth}{0pt} % убрать разделительную линию
\pagestyle{fancy}

% Pictures (plots ect)
\usepackage{pgfplots} 
%% Export to CSV
%\usepackage{filecontents}
%\begin{filecontents*}{data.csv}
%n,s1,s2,s3,s4
%5,0.00001168,0.00001121,0.00002050,0.00001121
%
%\end{filecontents*}
\usepackage[raggedright]{titlesec}
% Управление содержанием
\usepackage{titletoc}
\titleformat{\chapter}[display] % раздел
	{\filcenter}	% центровать
    %{\MakeUppercase{\chaptertitlename} \thechapter}	% РОЗДІЛ Х 
    {}
    {0pt} % отступ после
	{\MakeUppercase}	% название капсом
	{}
\assignpagestyle{\chapter}{fancy}
\titlespacing{\chapter} % указуємо, що модифікуємо саме розділ
	{0ex} % відступ зліва
	{-30pt} % відступ зверху 
	{8pt} % відступ знизу
\titlecontents{chapter}
	[0ex] %
	{}
	{\MakeUppercase{\chaptername\ \thecontentslabel {}\quad}\MakeUppercase}
	{}
	{\dotfill\contentspage}    

%	ПІДРОЗДІЛ
\titleformat{\section}[block]
    {\normalsize}
    {\thesection\quad}
    {1em}
    {}
\titlespacing{\section}
	{\parindent}
	{0pt}
	{0pt}
\titlecontents{section}
	[5ex]
	{}
	{\thecontentslabel\quad}
	{}
	{\dotfill\contentspage}    

\usepackage{pgfplotstable}
\usepackage{csvsimple}
\usepackage{longtable}
\begin{document}
\chapter{Реализация алгоритмов}
\section{Еффективные алгоритмы реализации АТД сисок, стек, очередь}
\thispagestyle{empty}

Абстрактный тип данных (АТД)~--- это математическая модель для типов данных, где тип данных определяется поведением (семантикой) с точки зрения пользователя данных, а именно в терминах возможных значений, возможных операций над данными этого типа и поведения этих операций.

В программировании абстрактные типы данных обычно представляются в виде интерфейсов, которые скрывают соответствующие реализации типов. Программисты работают с абстрактными типами данных исключительно через их интерфейсы, поскольку реализация может в будущем измениться. Такой подход соответствует принципу инкапсуляции в объектно-ориентированном программировании. Сильной стороной этой методики является именно сокрытие реализации. 

Абстрактные типы данных позволяют достичь модульности программных продуктов и иметь несколько альтернативных взаимозаменяемых реализаций отдельного модуля.

Cписок~--- это абстрактный тип данных, представляющий собой упорядоченный набор значений, в котором некоторое значение может встречаться более одного раза. Экземпляр списка является компьютерной реализацией математического понятия конечной последовательности. Экземпляры значений, находящихся в списке, называются элементами списка; если значение встречается несколько раз, каждое вхождение считается отдельным элементом. В языке Lisp список~--- основная структура данный, как и во многих других функциональных языках.

Стек~--- абстрактный тип данных, представляющий собой список элементов, организованных по принципу LIFO (англ. last in — first out, <<последним пришёл — первым вышел>>).

Очередь~--- абстрактный тип данных с дисциплиной доступа к элементам <<первый пришёл~--- первый вышел>> (FIFO, First In — First Out). Добавление элемента (принято обозначать словом enqueue~---  поставить в очередь) возможно лишь в конец очереди, выборка~--- только из начала очереди (что принято называть словом dequeue~--- убрать из очереди), при этом выбранный элемент из очереди удаляется.

\section{Операции над АТД}
Со \textit{стеком} возможны три операции: добавление элемента (иначе проталкивание, push), удаление элемента (pop) чтение головного элемента (peek).
В \textit{очередь} можно добавить элемент (только в конец) и извечь элемент (только с начала).
В зависимости от вида и способа реализации \textit{списка} может быть или отсутствовать возможность иметь доступ к произвольному элементу и вставлять/удалять элемент в произвольную позицию списка.

\vfill
\begin{flushright}
Сенчишен Д.А.
\end{flushright}
\end{document}
